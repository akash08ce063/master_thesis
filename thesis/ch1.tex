\chapter{Introduction}

Due to the recent development of Graphics Process Units (GPUs) hardware architecture and the massive parallel programming model that enables developers to fully exploit the computation power of GPUs, parallel computing using GPU has become more and more popular to develop high performance applications. Realistic image synthesis, especially global illumination which is one of the most computationally complex algorithm, has drawn much research interests focusing on developing new parallelized approaches to unleash the computation power of GPU. Photon mapping is one of the most popular global illumination approaches these days. Many acceleration techniques developed for photon mapping can achieve very good performance for static scenes. However, dynamic scenes such as the scene in which moving lights sources may have a big impact on the performance of renderer. One of biggest causes of this overhead is that the the dynamic scene requires the reconstruction of the acceleration data structure every frame. The data structure for photon mapping is usually balanced kd-tree \cite{Bentley:1975:MBS:361002.361007} used to speed up the photon search. The parallelize kd-tree construction algorithm will greatly increase the peak memory consumption though it is a pretty fast algorithm. 

In this thesis we would like to present a novel approach based on the traditional photon mapping technique on GPU.  Using this approach we could rearrange the photons data to avoid the reconstruction of kd-tree for dynamic scenes with moving light sources, an incremental update algorithm is also employed so that we can achieve the same image quality with better performance. In order to proof our proposed technique in terms of speed, memory consumption and image quality, we will implement the prototype of the existing technique and compare it with the new approach. 

\section{Structure of the thesis} 

Following the introduction, we will firstly introduce the most important theoretical concepts and frameworks to establish a foundation for the subsequent discussion of global illumination approaches. Then we will make a short survey on various approaches and their strength and weakness when dealing with different phenomena. Afterwards we will highlight GPU-based techniques applied on Monte Carlo ray tracing and photon mapping and discuss some open issues of the current approach for the introduction of our new approach in the following chapter.  In chapter three, we presented a new GPU photon mapping approach with detailed information on our current implementation. The results of our experiment and the analysis of the results are presented in chapter four. Finally, in chapter five we come to the conclusion, and discuss the directions for future work.