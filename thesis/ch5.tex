\chapter{Conclusion}

\section{Conclusion}
In this research, we mainly focus on developing an improved photon mapping technique specifically for current generation GPU taking the advantage of the massive parallelized computation power. We started with introducing the theoretic basis of global illumination, then we reviewed the existing approaches published on the solving the problem, especially the photon mapping techniques exploited the parallelism with GPU. In our opinion, the existing method suffers from a deficiencies in how it handles the dynamic scene. After analyzing the weakness, we present a new approach trying to avoid another kd-tree construction exclusively for photon data and support an incremental updating scheme by associating the photons data with the kd-tree of the geometric objects and accumulating the photons from previous frames for k nearest neighbor photon searching. To proof our concept we implemented the test programs for both the existing and our new approach and carried out a series of benchmarks. 

In our tests we observed that our approach was faster during construction and almost all the photon search tests among certain range of frames for a dynamic scene. Especially the construction time was greatly reduced compared to the old approach. Along with speed measurements, we also examined the memory consumption of both approaches. Our approach consumes more memory in rendering phase since the photon data from previous frames. But the memory consumption in the construction phase is much less than the existing method. Because we only require the construction of kd-tree for the scene. The memory consumption also strongly depends on the number of frames we want to store for the photons. Finally, we validate the quality of the rendered by directly visualizing the photons in the scene. 

\section{Future Work}

We believe that there are a couple of interesting topics the future work can work on to improve our new approach. The first one is improve the algorithm of the classifying the photons to the kd-tree's leaf nodes to achieve more efficient parallelization. The current solution is that each thread works on one leaf node iterating over all the photons, this could lead considerable deficiency of parallelization on GPU especially when the height of the tree is relative low (there are less leaf nodes) due to the low occupancy of CUDA threads. One possible solution is to map the photons to CUDA threads instead, mark the photons that belong  to certain kd-tree leaf node and apply parallel primitive algorithm such as compact to calculate the indices of photons for each kd-tree leaf node in parallel. 

Another interesting filed we can explore further is to test and observe that how participating media such as smoke, dust or clouds effect the performance using our approach. The participating media will affect light when it travels through them, the light beam is either absorbed or scattered. Since our approach spatially encodes the distribution of the photons based on the volume boundary of the kd-tree nodes, we think our approach is suitable for storing photon information of volumetric participating media and good performance could be expected. 

In addition to keep evolving the approach algorithmically, we certainly cannot ignore the impact brought by the developments in graphics hardware on the photon mapping techniques. With the latest generation of GPUs and development framework, the applications would be beneficial from better better performance and more flexibility, such as better performance for uncoalesced memory access implicit hardware optimizations which is almost free for developers. 